\documentclass[10pt]{article}
\usepackage{amsmath}
\usepackage{amssymb}
\usepackage{framed}
\usepackage{graphicx}
\usepackage{cite}

\topmargin 0.0cm
\oddsidemargin 0.5cm
\evensidemargin 0.5cm
\textwidth 16cm 
\textheight 21cm

\usepackage[labelfont=bf,labelsep=period,justification=raggedright]{caption}

\bibliographystyle{plos2009}
\makeatletter
\renewcommand{\@biblabel}[1]{\quad#1.}
\makeatother

\date{}

\pagestyle{myheadings}

\newcommand{\fixme}[1]{\textsc{\textbf{#1}}}
\newcommand{\recommend}[1]{\textit{#1}}
\newcommand{\withurl}[2]{{#1}\footnote{\texttt{#2}}}

\begin{document}

\begin{flushleft}
{\Large
\textbf{Good Enough Practices for Scientific Computing}
}

{Greg~Wilson}$^{1,\ast}$,
{Jenny~Bryan}$^{2}$,
{Karen~Cranston}$^{3}$,
{Justin~Kitzes}$^{4}$,
{Lex~Nederbragt}$^{5}$,
{Tracy~K.~Teal}$^{6}$
\\
1) Software Carpentry Foundation / gvwilson@software-carpentry.org
\\
2) University of British Columbia / jenny@stat.ubc.ca
\\
3) Duke University / karen.cranston@duke.edu
\\
4) University of California, Berkeley / jkitzes@berkeley.edu
\\
5) University of Oslo / lex.nederbragt@ibv.uio.no
\\
6) Data Carpentry / tkteal@datacarpentry.org
\\
$\ast$ E-mail: Corresponding gvwilson@software-carpentry.org
\end{flushleft}

\section{Introduction}\label{sec:introduction}

Two years ago a group of researchers involved in \withurl{Software
  Carpentry}{http://software-carpentry.org/} and \withurl{Data
  Carpentry}{http://datacarpentry.org/} wrote a paper called
``\withurl{Best Practices for Scientific
  Computing}{http://journals.plos.org/plosbiology/article?id=10.1371/journal.pbio.1001745}''
\cite{wilson2014}. It was well received,
but many novices found its catalog of tools and
techniques intimidating, and by definition, the ``best'' are a
minority, so ``best practices'' are what only a minority does.

This paper therefore presents a set of ``good enough''
practices\footnote{Note that English lacks a good word for this:
  ``mediocre'', ``adequate'', and ``sufficient'' aren't exactly
  right.} for scientific computing, i.e., a minimum set of tools and techniques that we believe every
researcher can and should adopt. It draws inspiration from many
sources \cite{gentzkow2014,noble2009,brown2015,wickham2014,kitzes2016,sandve2013,hart2015},
from our personal experiences,
and from the experiences of the thousands of people who have taken part
in \withurl{Software Carpentry}{http://software-carpentry.org}
and \withurl{Data Carpentry}{http://datacarpentry.org} workshops over the past six years.

Our intended audience is researchers who are working alone or with a
handful of collaborators on projects lasting a few days to a few
months, and who are ready to move beyond saving spreadsheets
with names like \texttt{results-updated-3-revised.xlsx} in Dropbox. A practice
is included in our list if large numbers of researchers use it, and large numbers
of novices are \emph{still} using it
months after first trying it out. We include the second criterion because
there is no point recommending something that people
abandon after a few days or weeks because they don't
perceive value.

Many of our recommendations are
for the benefit of the collaborator every researcher cares about most:
their future self\footnote{As the joke goes, yourself from three months ago doesn't
  answer email{\ldots}}. If researchers don't see those benefits
quickly enough to pay for the slowdown that inevitably occurs when
adopting something new, they will almost certainly switch
back to their old way of doing things.
This rules out many practices
such as code review that we feel are essential for larger-scale
development (Section~\ref{sec:omitted}).

\fixme{add pointers to papers or projects that follow these rules.}

\section{Data Management}\label{sec:data}

Project data may need to exist in various forms, ranging from what
first arrives to what's included in the final publication.  The aim
of our recommendations is to create ``tidy data'', which can be
a powerful accelerator for analysis \cite{wickham2014,hart2015}.

\begin{enumerate}

\item
  \recommend{Save data in the rawest form available}.  Where possible,
  save the data file produced by an instrument or raw results from a
  survey, with all of its mystifying imperfections (e.g., raw JPEGs
  for photographs). It is tempting to overwrite raw data files with
  cleaned-up versions, but faithful retention is essential to being
  able to re-run analyses from start to finish. It also
  helps you recover from
  analytical mishaps and allows you to experiment without fear.

  But use common sense: if a large volume of data is received in a
  format that is inefficient to store or process,
  transform it for storage with a lossless, well-documented
  procedure.  Equally, don't duplicate contents of stable, long-lived
  repositories (i.e., don't clone GenBank)

  You may also consider restricting file permissions to read-only,
  even (or especially) for yourself, so it is harder to damage raw
  data by accident or to hand edit it in a moment of weakness.

\item
  \recommend{Create the data you wish to see in the world}, i.e.,
  create the dataset you \emph{wish} you had received from the raw
  data and use that as the starting point for downstream analyses.
  This is the
  time to improve machine and human readability, but \emph{not}
  the time to do vigorous data filtering or bring in external information.

  Approach this initial tidying up as internal to the existing dataset
  and non-destructive, both of the data and its general
  ``shape''. 
  Where possible, use open non-proprietary formats instead of closed,
  proprietary formats: they will probably last longer, and will
  probably be easier for machines to read.
  CSV is good for simple tabular data,
  JSON, YAML, or XML for non-tabular data such as graphs\footnote{The
  node-and-arc kind.},
  and HDF5 for certain kinds of structured data.
  \withurl{This
    guide}{http://www.library.illinois.edu/sc/services/data\_management/file\_formats.html}
  has a few useful guidelines for choosing a file format.

  Enhance
  human readability by replacing inscrutable variable names and
  artificial data codes with self-explaining alternatives, e.g.,
  rename variables called \texttt{name1} and \texttt{name2} to
  \texttt{personal\_name} and \texttt{family\_name}, recode the
  treatment variable from \texttt{1} vs.  \texttt{2} to
  \texttt{untreated} vs. \texttt{treated}, and replace artificial
  codes for missing data, such as ``-99'', with proper \texttt{NA}s
  \cite{white2013}.

  Both human and machine readability can be enhanced by storing
  especially useful metadata as part of the filename itself, while
  keeping the filename regular enough for easy pattern matching. For
  example, a filename like \texttt{2016-05-alaska-b.csv} makes it easy
  for both people and programs to select by year
  (\texttt{2016-*.csv}), by location (\texttt{*-alaska-*.csv}), and so
  on.

\item
  \recommend{Create analysis-friendly data}.  You can do this by
  reformatting data to eliminate processing steps between loading the
  data set and doing the analysis.
  Columns that contain more than one variable's worth of information
  should be split. For example, the presence of ``kg'' in ``3.4 kg''
  will cause most analytical environments to read this in as character
  data rather than numeric. It should be split into two columns (the
  mass ``3.4'' and the units ``kg''), or the units should be recorded
  in the variable name and/or metadata.

  Multiple columns that only contain one variable's worth of information
  when taken together should be combined. This is characteristic of data
  that has been laid out for human eyeballs or for manual data entry.
  For example, there might be one row per field site and then columns
  for measurements made at each of several time points. It is convenient
  to store this in a ``short and wide'' form for data entry and
  inspection, but for most analyses it will be advantageous to gather
  these columns into a variable of measurements, accompanied by a
  companion variable indicating the time point.

  Reformat values as needed to match your environment's built-in parsing
  rules. For example, use date-time format that will be recognized
  automatically.

  \begin{framed}
    \noindent \textbf{OpenRefine}

    \withurl{OpenRefine}{http://openrefine.org/} is an excellent tool for
    this stage of data cleanup. It combines a spreadsheet-like interface
    to tabular data with a large set of cleanup heuristics, and can
    generate a trace of cleanup steps to ensure reproducibility.
  \end{framed}

\item
  \recommend{Always record the steps used to clean up data}:
  it's as much a part of your analysis as any statistics you might do.
  The best way to do this is to write a re-runnable script for \emph{every}
  stage of data cleanup, however trivial it seems.
  This not only ensures that you have a record,
  but also allows you to re-do the cleanup when new data arrives.

\item
  \recommend{Give every record a unique key}.  In a
  well-organized data set, it's easy to select a specific record to
  update or delete.  Using fields in the data (such as people's names or
  addresses) does not guarantee this; using row numbers in tables does,
  but those numbers will change as records are added, removed, or moved.

  We therefore recommend giving each record a unique, persistent key
  that has no other purpose than identifying the record\footnote{Student
    numbers and similar identifiers serve this purpose in institutional
    databases.} The provision of such a key also makes it much easier to
  link data sets together.

\item
  \recommend{Marshal complementary data in files that meet the
    same standards}.  Raw data frequently does not fully explain
  itself. For example, if each well of a microtitre plate is used to
  study a specific gene knockout, the prepared raw data will have a
  \texttt{well} variable taking on values like ``A1'' and ``G12'', but
  the plate reader cannot possibly know what's in each well. That
  information will be absent from the primary data file, so you must
  create a supplementary file in order to look this up. Make sure to use
  the same names and codes when variables in two datasets refer to the
  same thing to simplify merging and lookup (and make them less
  error-prone).

\item
  \recommend{Create a dataset purpose-built for specific analyses
    and figures.}  This probably involves filtering rows, selecting
  relevant variables, and merging with external information. If the
  preceding steps have been done, this can often be surprisingly
  straightforward. The basic form of the data was hopefully set earlier,
  so the main changes here are likely to be data reduction and the
  amalgamation of multiple datasets.

\item
  \recommend{Represent every analysis step textually, complete with
    parameter values.}  It isn't always possible to store the steps
  themselves textually, but it is always possible to store the result
  for later checking.  For example, choosing a region of intest in an
  image is inherently interactive, but the region chosen can be stored
  as a set of boundary coordinates.

\item
  \recommend{Submit data to a reputable DOI-issuing
  repository so that others can access and cite it.}  Your data is as
  much a product of your research as the papers you write, and just as
  likely to be useful to others (if not more so).

\end{enumerate}

Taken in order, the recommendations above will
produce intermediate data files with increasing levels of cleanliness
and task-specificity. This makes it easy to re-run \emph{parts}
of a data analysis pipeline, which in turn makes it easier and faster
to tinker with specific data cleaning operations.

We propose delimited plain text (such as CSV) as a lowest common
denominator for data that is usable across time, people,
operating systems, and analytical environments. However,
plain text is not a panacea: when collaborating with
others, be aware of and, when possible, standardize on file encoding,
line endings, and the elimination of prose-oriented features, such as
``smart quotes''. Differences between collaborators in these
details can cause the changes between file versions to be large, and
therefore substantially less comprehensible.

\begin{framed}
\noindent \textbf{Choosing Metadata}

\cite{wickes2015} provides a useful classification for metadata, with implications for
how to represent it. She notes it is important to distinguish
metadata about the dataset as a whole (e.g., author and related publications)
from metadata about the content (e.g., units for observations).
When considering how to store metadata, consider the intended
audience. If it is human beings, write a README file. If it is machines, such as
metadata harvesters and formal repositories, create a metadata file
that they will easily be able to parse.
\end{framed}

\section{Software}\label{sec:software}

If you or your group are creating tens of thousands of lines of software
for use by hundreds of people you have never met,
you are doing software engineering. If you're writing a few dozen lines now
and again, and are probably going to be its only user, you may not be doing engineering,
but you can still
make things easier on yourself by adopting a few key engineering practices. What's
more, adopting these practices will make it easier for people to
understand and (re)use your code.

The core realization in these practices is that \emph{readable},
\emph{reusable}, and \emph{testable} are
all side effects of writing \emph{modular} code, i.e., of building programs out
of short, single-purpose functions with clearly-defined inputs and
outputs.

\begin{enumerate}

\item  
  \recommend{Place a brief explanatory comment at the start of
    every program}, no matter how short it is. That comment should
  include at least one example of how the program is used: remember, an
  example is worth a thousand words. Where possible, the comment should
  also indicate reasonable values for parameters.  Figure~\ref{fig:comment}
  presents an example of of such a comment.

\item
  \recommend{Decompose programs into functions} that
  are no more than one page long (i.e., 60 lines including spaces and comments),
  do not use global variables (constants are OK),
  and take no more than half a dozen parameters.
  The key motivation here is to fit the program into the most limited
  memory of all: ours. Human short-term memory is famously incapable of
  holding more than about seven (plus or minus two) items at once. If we
  are to understand what our software is doing, we must break it into
  chunks that obey this limit, then create programs by composing chunks
  into larger chunks and so on.

\item
  \recommend{Be ruthless about eliminating duplication}. Write and
  re-use functions instead of copying and pasting source code, and use
  data structures like lists rather than creating lots of variables called
  \texttt{score1}, \texttt{score2}, \texttt{score3}, etc.

  This applies with even greater force to using libraries. The easiest code
  to debug and maintain is code you didn't actually write, so
  always search for libraries that do what you need before writing
  new code yourself.

\item
  \recommend{Give functions and variables meaningful names}, both to
  document their purpose and to make the program easier to read. As a rule
  of thumb, the greater the scope of a variable, the more informative its
  name should be: while it's acceptable to call the counter variable in a
  loop \texttt{i} or \texttt{j}, the main grid for your simulation should
  \emph{not} have a one-letter name.

  \begin{framed}
    \noindent \textbf{Tab Completion}

    Almost all modern text editors provide \emph{tab completion},
    so that typing the first part of a variable name and then pressing the tab key
    inserts the completed name of the variable.
    Employing this means that meaningful variable names are no harder to type than short ones.

  \end{framed}

\item
  \recommend{Make dependencies and requirements explicit.} This is
  usually done on a per-project rather than per-program basis, i.e., by
  adding a file called something like \texttt{requirements.txt} to the
  root directory of the project, or by adding a ``Getting Started''
  section to the \texttt{README} file. (More sophisticated users may
  structure the requirements in a way that installation management tools
  can use.)

\item
  \recommend{Do not comment and uncomment sections of code to control
    a program's behavior}, since this is error prone and makes it difficult
  or impossible to automate analyses. Instead, put if/else statements in
  the program to control what it does.

  \begin{framed}
    \noindent \textbf{Logging}

    One special case of this recommendation is to use a logging library
    such as Java's \texttt{log4j} for debugging messages. When this is
    done, messages are left in the code, but their activity is controlled
    by an external configuration file.
  \end{framed}

\item
  \recommend{Provide a simple example or test data set} that users
  (including yourself) can run to determine whether the program is working
  at all and whether it gives a known correct output for a simple known
  input. Such as ``build and smoke test'' is particularly helpful when
  supposedly-innocent changes are being made to the program, or when it
  has to run on several different machines, e.g., the developer's laptop
  and the department's cluster.

\item
  \recommend{Submit code to a reputable DOI-issuing repository}
  upon submission of paper, just as you do with data. Your software is as
  much a product of your research as your papers, and should be as easy
  for people to credit.

\end{enumerate}

\begin{figure}
\begin{verbatim}
Synthesize image files for testing circularity estimation algorithm.

Usage: generate_images.py -d p_diameter -f p_flaws -r rand_seed -s image_size

where:
-f fuzzing     = fuzzing range of blobs (typically 0.0-0.2)
-n p_flaws     = p(success) for geometric distribution of # flaws/sample (typically 0.5-0.8)
-o output_file = name of output file
-r p_radius    = p(success) for geometric distribution of flaw radius (typically 0.1-0.4)
-s rand_seed   = RNG seed (large integer)
-v             = verbose
-w size        = image width/height in pixels (typically 480-800)
\end{verbatim}
\caption{A Good Explanatory Comment}
\label{fig:comment}
\end{figure}

\section{Collaboration}\label{sec:collaboration}

You may start working on projects by yourself or with a small group of
collaborators you already know, but you should still design it to
attract new collaborators.  As summarized in \cite{steinmacher2015},
the keys to doing this are to make it easy for people set up a local
workspace so that they \emph{can} contribute, to help them find a
simple first task so that they know \emph{what} to contribute, and
to make the contribution process clear so that they know \emph{how}
to contribute.

\begin{enumerate}

\item
  \recommend{Create an overview of your project.}  Have a short
  \texttt{README} file explaining the project's purpose in the
  project's home directory.  This file (which may be called
  \texttt{README}, \texttt{README.txt}, or something similar) should
  contain the project's title, a brief description (similar to the
  abstract of a paper), and contact information that actually works.
  It is often the first thing users of your project will look at, so
  make it explicit that you welcome contributors and point them to
  ways they can help.
  
  You should also create a \texttt{CONTRIBUTING} file that describes
  what people need to do in order to get the project going and
  contribute to it, i.e., dependencies that need to be installed,
  tests that can be run to ensure the software has been installed
  correctly, and guidelines or checklists that your project adheres
  to.

\item
  \recommend{Create a shared public ``to-do'' list}.  This can be a
  plain text file called something like \texttt{notes.txt} or
  \texttt{todo.txt}, or you can use sites such as GitHub or Bitbucket
  to create a new \emph{issue} for each to-do item. (You can even add
  labels such as ``low hanging fruit'' to point newcomers at issues
  that are good starting points.)  Whatever you choose, describe the
  items clearl so that they make sense to newcomers.

\item
  \recommend{Make the license explicit.}  Have a \texttt{LICENSE}
  file in the project's home directory that clearly states what
  license(s) apply to the project's software, data, and
  manuscripts. Lack of an explicit license does not mean there isn't
  one; rather, it implies the author is keeping all rights and others
  are not allowed to re-use or modify the material.

  We recommend Creative Commons licenses for data and text, either
  \withurl{CC-0}{https://creativecommons.org/about/cc0/} (the ``No
  Rights Reserved'' license) or
  \withurl{CC-BY}{https://creativecommons.org/licenses/by/4.0/} (the
  ``Attribution'' license, which sharing and reuse but requires people
  to give appropriate credit to the creators).  For software, we
  recommend a permissive license such as the MIT, BSD, or Apache
  license \cite{laurent2004}.

  \begin{framed}
    \noindent \textbf{What Not To Do}

    We recommend \emph{against} the ``no commercial use'' variations of
    the Creative Commons licenses because they may impede some forms of
    re-use.  For example, if a researcher in a developing country is being
    paid by her government to compile a public health report, and wishes
    to include some of your data, she will be unable to do so if the
    license says ``non-commercial''. We recommend permissive software
    licenses rather than the \withurl{GNU General Public
      License}{https://www.safaribooksonline.com/library/view/understanding-open-source/0596005814/ch03.html}
    (GPL) because it is easier to integrate permissively-licensed software
    into other projects. (Note that it is straightforward to switch from a
    permissive license to the GPL if you should change your mind later but
    rather more complicated to go in the other direction.)
  \end{framed}

\item    
  \recommend{Make the project citable} by including a
  \texttt{CITATION} file in the project's home directory that
  describes how to cite this project as a whole, and where to find,
  and how to cite, any data sets, code, figures, and other artifacts
  that have their own DOIs.  Figure~\ref{fig:citation} shows the
  \texttt{CITATION} file for the Software Carpentry project.
  For an example of a more detailed \texttt{CITATION} file, see the one
  for the \withurl{khmer project}{https://github.com/dib-lab/khmer/blob/master/CITATION}.

\end{enumerate}

The overview and to-do list are to help you as well as other
people---remember, your most important collaborator is yourself three
months from now. The license and citation file are there to make it
easy for other people to help you and give you credit for your work.

\begin{figure}
\begin{verbatim}
To reference Software Carpentry in publications, please cite:

Greg Wilson: "Software Carpentry: Lessons Learned". F1000Research,
2016, 3:62 (doi: 10.12688/f1000research.3-62.v2).

@online{wilson-software-carpentry-2016,
  author      = {Greg Wilson},
  title       = {Software Carpentry: Lessons Learned},
  version     = {2},
  date        = {2016-01-28},
  url         = {http://f1000research.com/articles/3-62/v2},
  doi         = {10.12688/f1000research.3-62.v2}
}
\end{verbatim}
\caption{Example CITATION File}
\label{fig:citation}
\end{figure}

\section{Project Organization}\label{sec:project}

Organizing the files that make up a project in a logical and
consistent directory structure will help you and others keep track of
them.  Our recommendations for doing this are drawn primarily from
\cite{noble2009,gentzkow2014}.

\begin{enumerate}

\item
  \recommend{Put each project in its own directory, which is named
    after the project.}  Like deciding when a chunk of code should be
  made a function, the ultimate goal of dividing research into
  distinct projects is to help you and others best understand your
  work. Some researchers create a separate project for each manuscript
  they are working on, while others group all research on a common
  theme, data set, or algorithm into a single project.
    
  As a rule of thumb, divide work into projects based on the overlap
  in data and code files. If two research efforts share no data or
  code, they will probably be easiest to manage independently. If they
  share more than half of their data and code, they are probably best
  managed together. Anything in between can be decided based on the
  set of people you're collaborating with.

\item
  \recommend{Put text documents associated with the projectin the
    \texttt{doc} directory.} This includes files for manuscripts,
  documentation for source code, and/or an electronic lab notebook
  recording your experiments.  Subdirectories may be created for these
  different classes of files in large projects.

\item
  \recommend{Put raw data and metadata in a \texttt{data} directory,
    and files generated during cleanup and analysis in a
    \texttt{results} directory}, where ``generated files'' includes
  intermediate results, such as cleaned data sets or simulated data,
  as well as final results such as figures and tables.

  The \texttt{src} directory may contain subdirectories if there are
  many data sets.  The \texttt{results} directory will \emph{usually}
  require additional subdirectories for all but the simplest
  projects. Intermediate files such as cleaned data, statistical
  tables, and final publication-ready figures or tables should be
  separated clearly by file naming conventions or placed into
  different subdirectories; those belonging to different papers or
  other publications should be grouped together.

\item
  \recommend{Put source for this project's scripts and programs in the
    \texttt{src} directory, and scripts and programs brought in from
    elsewhere or compiled locally in the \texttt{bin} directory.}
  \texttt{src} contains both programs written in interpreted languages
  such as R or Python and those written compiled languages like
  Fortran, C++, or Java. Shell scripts, snippets of SQL used to pull
  information from databases, and everything else needed to regenerate
  the results are all managed like source code.  \texttt{bin} contains
  scripts that are brought in from elsewhere, and executable programs
  compiled from code in the \texttt{src} directory\footnote{The name
    \texttt{bin} is an old Unix convention, and comes from the term
    ``binary''}. Projects that have neither will not require this
  directory.

  \begin{framed}
    \noindent \textbf{Scripts vs.\ Programs}

    We use the term ``script'' to mean ``something that is executed
    directly as-is'', and ``program'' to mean ``something that is
    explicitly compiled before being used''.  The distinction is more
    one of degree than kind---libraries written in Python are actually
    compiled to bytecode as they are loaded, for example---so one
    other way to think of it is ``things that are edited directly''
    and ``things that are not''.
  \end{framed}

\item
  \recommend{Name all files to reflect their content or function.} For
  example, use names such as \texttt{bird\_count\_table.csv},
  \texttt{manuscript.md}), or \texttt{sightings\_analysis.py}).  Do
  \emph{not} using sequential numbers (e.g., \texttt{result1.csv},
  \texttt{result2.csv}) or a location in a final manuscript (e.g.,
  \texttt{fig\_3\_a.png}), since those numbers will almost certainly
  change as the project evolves.
  
\end{enumerate}

The \texttt{src} directory may contain two conceptually distinct types
of files that should be distinguished either by clear file names or by
additional subdirectories. The first type are individual files or
related groups of files that contain functions to perform the core
analysis of the research. One file, for example, may contain functions
used for data cleaning, while another contains functions to do certain
statistical analyses. These files can be thought of as the
``scientific guts'' of the project, and as the project grows, they can
be organized into additional subdirectories. If a project were to
include re-runnable tests (Section~\ref{sec:omitted}), these are the
functions they would exercise.
  
The second type of file in \texttt{src} is controller or driver
scripts that combine the core analytical functions with particular
parameters and data input/output commands in order to execute the
entire project analysis from start to finish. A controller script for
a simple project, for example, may read a raw data table, import and
apply several cleanup and analysis functions from the other files in
this directory, and create and save a numeric result. For a small
project with one main output, a single controller script should be
placed in the main \texttt{src} directory and distinguished clearly by
a name such as ``runall''.

The controller script is the glue that holds the analysis together and
allows a single command, such as \texttt{python runall.py}, to re-run
the entire analysis from start to finish. These scripts should be
short (no more than one or two pages long) and very easy to
understand.  In particular, while they may contain loops (to process
multiple data files or sweep across parameter ranges) they should
contain few if any conditional statements or new function
definitions. If a control script becomes longer or more complicated
than this, or begins to include code that would require a new
collaborator more than a minute or two to understand, those portions
of the code should be moved out of the controller script and into
other core analysis files in this directory.

Figure~\ref{fig:layout} below provides a concrete example of how a
simple project might be organized following these recommendations. The
root directory contains a \texttt{README} file that provides an
overview of the project as a whole and a \texttt{CITATION} file that
explains how to reference it. The \texttt{data} directory contains a
single CSV file with tabular data on bird counts (machine-readable
metadata could also be included here). The \texttt{src} directory
contains \texttt{sightings\_analysis.py}, a Python file containing
functions to summarize the tabular data, and a controller script
\texttt{runall.py} that loads the data table, applies functions
imported from \texttt{sightings\_analysis.py}, and saves a table of
summarized results in the \texttt{results} directory.

This project doesn't have a \texttt{bin} directory, since it does not
rely on any compiled software. The \texttt{doc} directory contains two
text files written in Markdown, one containing a running lab notebook
describing various ideas for the project and how these were implemented
and the other containing a running draft of a manuscript describing the
project findings.

\begin{figure}
\begin{verbatim}
.
|-- CITATION
|-- README
|-- data/
|   \-- birds_count_table.csv
|-- doc/
|   |-- notebook.md
|   \-- manuscript.md
|-- results/
|   \-- summarized_results.csv
|-- src/
|   |-- sightings_analysis.py
|   \-- runall.py
\end{verbatim}
\caption{Example Project Layout}
\label{fig:layout}
\end{figure}

\section{Version Control}\label{sec:versioning}

Keeping track of changes that you or your collaborators make to data,
software, and manuscripts is a critical part of research.  We believe
that the best tools for doing this are the version control systems
that are used in software development, such as Git, Mercurial, and
Subversion.  They keep track of what was changed in a file when and by
whom, and synchronize changes to a central server so that many users
can track the same set of files.

All of the authors use version control daily for all of their
projects.  However, many competent and productive researchers who have
used version control in earnest believe that using it for managing
papers and data is like using the back end of a screwdriver to hammer
in a nail.  As Arjun Raj said in \withurl{a blog
  post}{http://rajlaboratory.blogspot.ca/2016/03/from-over-reproducibility-to.html}:

\begin{quote}
Google Docs excels at easy sharing, collaboration, simultaneous
editing, commenting and reply-to-commenting. Sure, one can approximate
these using text-based systems and version control. The question is
why anyone would like to{\ldots}

The goal of reproducible research is to make sure one
can{\dots}reproduce{\ldots}computational analyses. The goal of version
control is to track changes to source code. These are fundamentally
distinct goals, and while there is some overlap, version control is
merely a tool to help achieve that, and comes with so much overhead
and baggage that it is often not worth the effort.

I've heard some say some version of ``If you're not using version
control, it's not science.'' This assertion is of course patently
false, as plenty of people have been doing fine science without
version control. Widespread adoption will require practical,
non-ideological reasons why one should use version control, and some
consideration of the serious hassles associated with its use.
\end{quote}

We therefore recommend that projects adopt \emph{either} a systematic
manual approach for managing changes \emph{or} version control in its
full glory.  Whatever is adopted should aid reproducibility by
allowing you to reference or retrieve a specific version of the entire
project. This is valuable for your future self (when you finally get
the reviews back for your paper), for your lab-mates and collaborators
(in case you leave the project), and for reviewers, editors, and
others who want to convince themselves of the conclusions in your
published research.  It should also support sharing and collaboration
by managing the process of merging independent changes made by
different people, and distributing those changes back to everyone in a
controlled, traceable way.

Whatever is chosen, we recommend that it be used in the following way:

\begin{enumerate}

\item
  \recommend{Back up (almost) everything created by a human being as
    soon as it is created.} This includes scripts and programs of all
  kinds, software packages that your project depends on, and
  documentation. A few exceptions to this rule are discussed below.

\item
  \recommend{Keep changes small.}  Each change should not be so large
  as to make the change tracking irrelevant. For example, a single
  change such as ``Revise script file'' that adds or changes several
  hundred lines is likely too large, as it will not allow changes to
  different components of an analysis to be investigated
  separately. Similarly, changes should not be broken up into pieces
  that are too small, although we find that this is less of a danger
  with novices. As a rule of thumb, a good size for a single change is
  a group of edits that you could imagine wanting to undo in one step
  at some point in the future.

\item
  Everyone working on the project should \recommend{share changes
    frequently}, and incorporate changes from others just as
  frequently. Do not allow individual investigator's versions of the
  project repository to drift apart, as the effort required to merge
  differences goes up faster than the size of the difference. This is
  particularly important for the manual version control procedure
  describe above, which does not provide any assistance for merging
  simultaneous changes.

\item
  \recommend{Create, maintain, and use a checklist for committing and
    sharing changes to the project.}  The list should include
  writing commit messages that clearly explain any changes,
  the size and content of single commits,
  style guidelines for code,
  updating to-do lists,
  and bans on committing half-done work or broken code.
  See \cite{gawande2011} for more on the proven value of checklists.

\end{enumerate}

\subsection*{Manual Versioning}

Our first suggested approach, in which everything is done by hand, has
three parts:

\begin{enumerate}

\item
  \recommend{Store each project in a folder that is mirrored off the
    researcher's working machine} by a system such as Dropbox, and
  synchronize that folder at least daily. It may take a few minutes,
  but that time can be spent catching up on email, and is repaid the
  moment a laptop is stolen or its hard drive fails.

\item
  \recommend{Add a file called \texttt{CHANGELOG.txt} to the project's
    \texttt{docs} subfolder}, and make dated notes about changes to
  the project in this file in reverse chronological order (i.e., most
  recent first). This file is the equivalent of a lab notebook, and
  should contain entries like those shown in Figure~\ref{fig:changelog}.

\item
  \recommend{Copy the entire project whenever a significant change has
    been made} (i.e., one that materially affects the results being
  produced), and store that copy in a sub-folder whose name reflects
  the date in the area that's being synchronized. This approach
  results in projects being organized as shown in Figure~\ref{fig:manual}.
  Here, the \texttt{project\_name} folder is mapped to external storage
  (such as Dropbox), \texttt{current} is where development is done, and
  other folders within \texttt{project\_name} are old versions.

  \begin{framed}
    \noindent \textbf{Data is Cheap, Time is Expensive}

    Copying everything like this may seem wasteful, since many files won't
    have changed, but consider: a terabyte hard drive costs about \$50
    retail, which means that 50 GByte costs less than a latte. Provided
    large data files are kept out of the backed-up area (discussed below),
    this approach costs less than the time it would take to select files by
    hand for copying.
  \end{framed}

\end{enumerate}

\begin{figure}
\begin{verbatim}
## 2016-04-08

* Switched to cubic interpolation as default.
* Moved question about family's TB history to end of questionnaire.

## 2016-04-06

* Added option for cubic interpolation.
* Removed question about staph exposure (can be inferred from blood test results).
\end{verbatim}
\caption{Example Changelog}
\label{fig:changelog}
\end{figure}

\begin{figure}
\begin{verbatim}
.
|-- project_name/
|   \-- current
|       \-- ...project content as described earlier...
|   \-- 2016-03-01
|       \-- ...content of 'current' on Mar 1, 2016
|   \-- 2016-02-19
|       \-- ...content of 'current' on Feb 19, 2016
\end{verbatim}
\caption{Directory Hierarchy for Manual Versioning}
\label{fig:manual}
\end{figure}

This manual procedure satisfies the requirements outlined above without
needing any new tools. If multiple researchers are working on the same
project, though, they will need to coordinate so that only a single
person is working on specific files at any time. In particular, they may
wish to create one change log file per contributor, and to merge those
files whenever a backup copy is made.

\subsection*{Tool-Based Versioning}

What the manual process described above requires most is
self-discipline. The version control tools that underpin our second
approach---the one we all now use for our projects---don't just
accelerate the manual process: they also automate some steps while
enforcing others, and thereby require less self-discipline for more
reliable results.

\begin{framed}
  \noindent \textbf{How Version Control Works}

  A version control system stores snapshots of a project's files in a
  repository. Users can modify their working copy of the project at will,
  and then commit changes to the repository when they wish to make a
  permanent record and/or share their work with colleagues. The version
  control system automatically records when the change was made and by
  whom along with the changes themselves.
  
  Crucially, if several people have edited files simultaneously, the
  version control system will detect the collision and require them to
  resolve any conflicts before recording the changes. Modern version
  control systems also allow repositories to be synchronized with each
  other, so that no one repository becomes a single point of failure.
  Tool-based version control has several benefits over manual version
  control:

  \begin{itemize}
  \item
    Instead of requiring users to copy everything into subfolders, version
    control safely stores just enough information to allow old versions of
    files to be re-created on demand. This saves both space and time.
  \item
    Instead of relying on users to choose sensible names for backup
    copies, the version control system timestamps all saved changes
    automatically.
  \item
    Instead of requiring users to be disciplined about writing log
    comments, version control systems prompt them every time a change is
    saved. They also keep a 100\% accurate record of what was
    \emph{actually} changed, as opposed to what the user \emph{thought}
    they changed, which can be invaluable when problems crop up later.
  \item
    Instead of simply copying files to remote storage, version control
    checks to see whether doing that would overwrite anyone else's work.
    This turns out to be the key to supporting large-scale ad hoc
    collaboration.
  \end{itemize}
\end{framed}

\fixme{Recommendations for tool-based version control go here??}

\subsection*{When is Version Control Not Necessary?}

Despite the benefits of version control systems, some types of files
may \emph{not} be appropriate for version control. First, if a file is
small, placing it in a version control repository facilitates
reproducibility. However, today's version control systems are not
designed to handle megabyte-sized files, never mind gigabytes, though
support for them is emerging.

Second, raw data should not change, and therefore should not require
version tracking. Putting synthesized or modified data sets into
version control may not be necessary if you can re-generate these
files from raw data and data-cleaning scripts (which definitely
\emph{are} under version control!).

Third, some data formats are unfortunately not amenable to version
control, which is designed to work with plain text files such as
source code.  In particular, Microsoft Office files (like the
\texttt{.docx} files used by Word or the \texttt{.xlsx} files used by
Excel) can be stored in a version control system, but you won't be
able to see specific changes. Similarly, tabular data (such as CSV
files) can be put in version control, but changing the order of the
rows or columns will create a big change for the version control
system, even if the data itself has not changed.

\begin{framed}
  \noindent \textbf{Inadvertent Sharing}

  Researchers dealing with data subject to legal restrictions that
  prohibit sharing (such as medical data) should be careful not to put
  data in public version control systems. Some institutions may provide
  access to private version control systems, so it is worth checking with
  your IT department.
\end{framed}

Opinion is divided on whether the \texttt{results} directory and other
generated files such as figures should be placed under version
control. If we borrow conventions from software development (just as
we borrowed version control itself) the answer is no, but there are
some benefits to putting results under version control in data
analysis projects:

\begin{itemize}
\item
  It gives collaborators immediate access to current processed data,
  results, figures, etc., without needing to regenerate it all.
\item
  Version control facilitates \emph{diffing}, i.e., seeing the
  differences between old and new states of files. Diffs can be used to
  see the downstream effects of actions like upgrading a piece of
  software, refactoring a script, or starting with a slightly different
  dataset.
\end{itemize}

If results files are kilobytes or a few megabytes in size, we
therefore recommend keeping them under version control. Anything more
than this, and something else should be used for management.

\section{Manuscripts}\label{sec:manuscripts}

Gathering data, analyzing it, and figuring out what it means is the
first 90\% of any project; writing up is the other 90\%. While the
writing step is rarely addressed in discussions of scientific computing,
computing has changed the process of manuscript generation just as much
as it has changed the research itself.

A common practice is for the lead author to send successive versions of
a manuscript to coauthors to collect feedback, which is returned as
changes to the document, comments on the document, plain text in email,
or a mix of all three. The result is a lot of files to keep track of and
much tedious manual labor to merge all the comments to create a new
master version, at which point the process begins again.

Instead of an email-based workflow, we recommend mirroring good
practices for managing software and data, to make writing scalable,
collaborative, and reproducible. As with our recommendations for version
control in general, we suggest that groups choose one of two different
approaches for managing manuscripts. The goals of both alternatives are
to:

\begin{enumerate}
\item
  Ensure that text is accessible to yourself and others now and in the
  future by making a single master document that is available to all
  coauthors at all times.
\item
  Reduce the chances of work being lost or people overwriting each
  other's work.
\item
  Make it easy to track and combine contributions from multiple
  collaborators.
\item
  Avoid duplication and manual entry of information, particularly in
  constructing bibliographies, tables of contents, and lists.
\item
  Make it easy to regenerate the final published form (e.g., a PDF) and
  to tell if it is up to date.
\item
  Make it easy to share that final version with collaborators and to
  submit it to a journal.
\end{enumerate}

\begin{framed}
\noindent \textbf{The First Rule Is{\ldots}}

Which workflow is chosen is less important than having all authors agree
on one or the other \emph{before} writing starts. Make sure to also
agree on a single method to provide feedback, be it an email thread or
mailing list, an issue tracker (like the ones provided by GitHub and
Bitbucket), or some sort of shared online to-do list.
\end{framed}

Our first recommended approach has two parts.  First, \recommend{write
  manuscripts using online tools with rich formatting, change
  tracking, and reference management}, such as Google Docs.  Second,
\recommend{include a \texttt{PUBLICATIONS} file in the project's
  \texttt{doc} directory} containing metadata about each online
manuscript (e.g., their URLs). This is analogous to the \texttt{data}
directory, which might contain links to the location of the data
file(s) rather than the actual files.

We realize that in many cases, even this solution is asking too much
from those who see no reason to move forward from desktop GUI tools. To
satisfy them, the manuscript can be converted to a desktop editor file
format (e.g., Microsoft Word's \texttt{.docx} or LibreOffice's
\texttt{.odt}) after major changes, then downloaded and saved in the
\texttt{doc} folder. Unfortunately, this means merging some changes and
suggestions manually, as existing tools cannot always do this
automatically when switching from a desktop file format to text and back
(although \texttt{{[}pandoc{]}{[}pandoc{]}} can go a long way).

The second approach treats papers exactly like software, and has been
used by researchers in mathematics, astronomy, physics, and related
disciplines for decades.

First, \recommend{write the manuscript in a plain text format that
  permits version control} such as
\withurl{LaTeX}{http://www.latex-project.org/} or
\withurl{Markdown}{http://daringfireball.net/projects/markdown/}, and
then convert them to other formats such as PDF as needed using
scriptable tools like \withurl{Pandoc}{http://pandoc.org/}.  Second,
\recommend{include tools needed to compile manuscripts in the project
  folder} and keep them under version control just like tools used to
do simulation or analysis.

This approach re-uses the version control tools and skills used to
manage data and software, and is a good starting point for
fully-reproducible research. However, it requires all contributors to
understand a much larger set of tools, including markdown or LaTeX,
make, BiBTeX, and Git/GitHub.

\subsection*{Discussion}

An early draft of this paper recommended plain text in version control
as the best way to manage manuscripts, but several reviewers pushed back
forcefully. For example, Stephen Turner wrote:

\begin{quote}
{\ldots}try to explain the notion of compiling a document to an
overworked physician you collaborate with. Oh, but before that, you have
to explain the difference between plain text and word processing. And
text editors. And markdown/LaTeX compilers. And BiBTeX. And Git. And
GitHub. Etc. Meanwhile he/she is getting paged from the OR{\ldots}

{\ldots}as much as we want to convince ourselves otherwise, when you
have to collaborate with those outside the scientific computing bubble,
the barrier to collaborating on papers in this framework is simply too
high to overcome. Good intentions aside, it always comes down to ``just
give me a Word document with tracked changes,'' or similar.
\end{quote}

In keeping with our goal of recommending ``good enough'' practices, we
have therefore included online storage in something like Google Docs. We
still recommend \emph{against} traditional desktop tools like
LibreOffice and Microsoft Word because they make collaboration more
difficult than necessary:

\begin{itemize}
\item
  If the document lives online (e.g., in Google Docs) then everyone's
  changes are in one place, and hence don't need to be merged manually.
\item
  If the document lives in a version control system, it provides good
  support for finding and merging differences resulting from concurrent
  changes. It also provides a convenient platform for making comments
  and performing review.
\item
  Both of our recommendations clearly define the master document and
  allow everyone to contribute to it on an equal footing.
\end{itemize}

As an example of an implementation of these recommendations, this paper
was written using a central online repository (GitHub), the \emph{issue}
functionality for discussing the outline and text, and \emph{pull
requests} for reviewing the contributions from different authors,
including collecting comments and suggestions on them, and other
contributors before merging them in.

\subsection*{Supplementary Materials}

Supplementary materials often contain much of the work that went into
the project, such as tables and figures or more elaborate descriptions
of the algorithms, software, methods, and analyses. In order to make
these materials as accessible to others as possible, do not rely solely
on the PDF format, since extracting data from PDFs is notoriously hard.
Instead, we recommend separating the results that you may expect others
to reuse (e.g., data in tables, data behind figures) into separate,
text-format files. The same holds for any commands or code you want to
include as supplementary material: use the format that most easily
enables reuse (source code files, Unix shell scripts etc).

\section{What's Not on This List}\label{sec:omitted}

We have deliberately left many good tools and practices off our list,
including some that we use daily.

\begin{description}

\item[\textbf{Branches}]
A \emph{branch} is a ``parallel universe'' within a version control
repository. Developers create branches so that they can make multiple
changes to a project independently. They are central to the way that
experienced developers use systems like Git, but they add an extra layer
of complexity to version control for newcomers. Programmers got along
fine in the days of CVS and Subversion without relying heavily on
branching, and branching can be adopted without significant disruption
after people have mastered a basic edit-commit workflow.

\item[\textbf{Build Tools}] Tools like
  \withurl{Make}{https://www.gnu.org/software/make/} were originally
  developed to recompile pieces of software that had fallen out of
  date. They are now used to regenerate data and entire papers: when
  one or more raw input files change, Make can automatically re-run
  those parts of the analysis that are affected, regenerate tables and
  plots, and then regenerate the human-readable PDF that depends on
  them.  However, novices can achieve the same behavior by writing
  shell scripts that re-run everything; these may do unnecessary work,
  but given the speed of today's machines, that is unimportant for
  small projects.

\item[\textbf{Unit Tests}] A \emph{unit test} is a small test of one
  particular feature of a piece of software. Projects rely on unit
  tests to prevent \emph{regression}, i.e., to ensure that a change to
  one part of the software doesn't break other parts. While unit tests
  are essential to the health of large libraries and programs, we have
  found that they usually aren't compelling for solo exploratory
  work. (Note, for example, the lack of a \texttt{test} directory in
  Noble's rules \cite{noble2009}.)  Rather than advocating something
  which people are unlikely to adopt, we have left unit testing off
  this list.

\item[\textbf{Continuous Integration}] Tools like
  \withurl{Travis-CI}{https://travis-ci.org/} automatically run a set of
  user-defined commands whenever changes are made to a version control
  repository. These commands typically execute tests to make sure that
  software hasn't regressed, i.e., that things which used to work
  still do. These tests can be run either before the commit takes
  place (in which case the changes can be rejected if something fails)
  or after (in which case the project's contributors can be notified
  of the breakage). CI systems are invaluable in large projects with
  many contributors, but pay fewer dividends in smaller projects where
  code is being written to do specific analyses.

\item[\textbf{Profiling and Performance Tuning}]
\emph{Profiling} is the act of measuring where a program spends its
time, and is an essential first step in \emph{tuning} the program (i.e.,
making it run faster). Both are worth doing, but only when the program's
performance is actually a bottleneck: in our experience, most users
spend more time getting the program right in the first place.

\item[\textbf{Coverage}]
Every modern programming language comes with tools to report the
\emph{coverage} of a set of test cases, i.e., the set of lines that are
and aren't actually executed when those tests are run. Mature projects
run these tools periodically to find code that isn't being used any
more, but as with unit testing, this only starts to pay off once the
project grows larger, and is therefore not recommended here.

\item[\textbf{The Semantic Web}] Ontologies and other formal
  definitions of data are useful, but in our experience, even
  simplified things like \withurl{Dublin Core}{http://dublincore.org/}
  are rarely encountered in the wild.

\item[\textbf{Data Structures}]
Associative data structures (e.g., Python's dictionaries), graphs (of
the node-and-arc kind), Bloom filters, and many other data structures
are essential to making programs both correct and efficient, but there
are too many, and they are too specific to particular problems, for us
to single any out for this list.

\item[\textbf{Documentation}]
Good documentation is a key factor in software adoption, but in
practice, people won't write comprehensive documentation until they have
collaborators who will use it. They will, however, quickly see the point
of a brief explanatory comment at the start of each script, so we have
recommended that as a first step.

\item[\textbf{A Bibliography Manager}] Researchers should use a
  reference manager of some sort, such as
  \withurl{Zotero}{http://zotero.org/}, and should also obtain and use an
  \withurl{ORCID}{http://orcid.org/} to identify themselves in their
  publications, but discussion of those is outside the scope of this
  paper.

\item[\textbf{Code Reviews and Pair Programming}]
These practices are valuable in projects with multiple contributors, but
are hard to adopt in single-author/single-user situations, which
includes most of the intended audience for this paper.

\end{description}

One important observation about this list is that many experienced
programmers actually do some or all of these things even for small
projects. It makes sense for them to do so because (a) they've already
paid the learning cost of the tool, so the time required to implement
for the ``next'' project is small, and (b) they understand that their
project will need some or all of these things as it scales, so they
might as well put it in place now.

The problem comes when those experienced developers give advice to
novices who \emph{haven't} already mastered the tools, and \emph{don't}
realize (yet) that they will save time if and when their project grows.
In that situation, advocating unit testing with coverage checking and
continuous integration is as likely to scare novices off than to aid
them.

\section{Conclusion}\label{sec:conclusion}

\fixme{write a conclusion}

\bibliography{good-enough-practices-for-scientific-computing}

\end{document}
